\documentclass[a4paper, 1pt]{article}
    
    % LANGUAGE PACKAGES
\usepackage[T1]{fontenc}
\usepackage[utf8]{inputenc}
\usepackage[brazil]{babel}


	% OTHER PACKAGES
\usepackage[margin=1in]{geometry}
\usepackage[none]{hyphenat}
\usepackage{fancyhdr}
\usepackage{capt-of}
\usepackage{graphicx,calc}
\usepackage{graphics}
\usepackage{float}
\usepackage{indentfirst}

% HEADER AND FOOOTER
\pagestyle{fancy}
\fancyhead{}
\fancyfoot{}
\fancyhead[L]{\slshape \MakeUppercase{Projeto SD}}
\fancyhead[R]{\slshape Sistemas Digitais 2021.2}
\fancyfoot[C]{\thepage}
\setlength{\headheight}{52pt}
\renewcommand{\footrulewidth}{0pt}

\begin{document}
\begin{titlepage}
\begin{center}
\vspace*{1cm}
\Large{\textbf{Sistemas Digitais 2021.2}\\Professor: Stephan Michael Blawid}\\
\vfill
\line(1,0){400}\\[1mm]
\huge{\textbf{Projeto 2}}\\[3mm]
\Large{Relatório do projeto proposto pela disciplina}\\[1mm]
\line(1,0){400}
\vfill
\begin{figure}[!htb]
    \centering
    \includegraphics[scale=0.2]{imgs/logo cin-ufpe.png}\\
    \label{fig:logo}
\end{figure}
Por Yves Emmanuel, David Londres, Natan Frederico e Clesson Roberto\\27 de Agosto de 2021
\end{center}
\end{titlepage}


\setcounter{page}{1}

\begin{section}{Primeira Questão}

\textbf{.A)} Os blocos funcionais do nível 2 são:  timer de minutos/segundos, entrada de timer e controle, controle do magnetron, decodificador/driver de 7 seguimentos.

\textbf{.B)} O \emph{clock} para o temporizador deve ser uma onda de 1 Hz, quando nenhum botão está sendo pressionado.

\textbf{.C)} Quando qualquer botão estiver sendo pressionado, então um sinal de 100 Hz deve ser direcionado para o temporizador.
\end{section}


\end{document}